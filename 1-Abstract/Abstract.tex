\prechapter{Abstract}

% Introduction to mass and radius of low-mass stars
The absolute parameters of M-dwarfs in eclipsing binary systems provide important tests for evolutionary models. Those that have been measured have revealed significant discrepancies with evolutionary models. There are two problems with M-dwarfs: 1. M-dwarfs generally appear bigger and cooler than models predict (such that their luminosity agrees with models) and 2. some M-dwarfs in eclipsing binaries are measured to be hotter than expected for their mass. The exact cause of this is unclear and a variety of conjectures have been put forward including enhanced magnetic activity and spotted surfaces. However, there is a lack of M-dwarfs with absolute parameters and so the exact causes of these disparities are unclear. As the interest in low-mass stars rises from the ever increasing number of exoplanets found around them, it is important that a considerable effort is made to understand why this is so.


% The EBLM project
A solution to the problem lies with low-mass eclipsing binary systems discovered by the WASP project. A large sample these systems have been followed up with spectroscopic orbits that ultimately exclude them from the planet-hunting process. In this work I obtained follow-up photometry for 9 of these systems and used these data to measure the absolute parameters of each star. These will eventually be used to create empirical calibrations for low mass stars when the number of EBLMs measured within this framework increases.

% Spectroscopic analysis of EBLM systems
Breaking the mass degeneracy required supplementary information from evolutionary models and the primary stars atmospheric parameters. I successfully created, tested and deployed a spectral analysis routine which used wavelet decomposition to analyse the spectra of FGK stars in exoplanet/eclipsing binary systems. Careful selection of wavelet coefficients filter out large systematic trends and noise typically observed in spectra. I used this principle to reliably measure $T_{\rm eff}$, $V \sin i$ and [Fe/H] from CORALIE spectra. My method had a systematic offset in [Fe/H] of $-0.18$\,dex relative to equivalent-width measurements of higher-quality spectra. There is also a trend between $T_{\rm eff}$ and $\log g$ which has unclear origins.


% Problems with he sample
The sample of eclipsing binary systems in this work highlight that only a fraction are suitable for empirical calibrations. I found that four of systems have primary stars which have evolved into the ``blue-hook'' part of their main-sequence evolution. They have two distinct solutions for mass and age which require supplementary information before they can be used in empirical calibrations. A further two systems have large impact parameters which increase the uncertainty in radius above the required precision of a few percent. I advocate the need for a volume-limited sample to avoid spending time observing and measuring such systems.

% Problems with the method
The method used to measure low-mass eclipsing binaries is well-established, yet there is a dearth well-studied F+M binaries. The EBLM project has provided spectroscopic orbits for 118 F+M binaries and I expect the absolute parameters for these systems to follow timely. However, there is a requirement for a \textit{hare-and-hounds} style experiment to assess how absolute parameters differ between different research groups and methods of analysis. I show that subtle choices in helium-enhancement and mixing-length parameters can introduce a 2-4\% uncertainty in mass and age. A similar effect is seen for different limb-darkening laws and so an in-depth review into how this will affect empirical mass-radius calibrations is required. 
